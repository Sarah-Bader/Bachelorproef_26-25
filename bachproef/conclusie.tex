%%=============================================================================
%% Conclusie
%%=============================================================================

\chapter{Conclusie}%
\label{ch:conclusie}

Het verwachte resultaat van deze studie is een proof of concept (PoC), die in staat is een subset van representatieve HLASM- en labels om te zetten in visuele stappen geïnterpreteerd in Mermaid.js-syntax. De PoC brengt door middel van statische HLASM-broncodeanalyse de control flow in kaart. 
Met deze PoC wordt verwacht beginnende mainframeontwikkelaars te ondersteunen met het begrijpen van een onbekend stuk HLASM-broncode. De evaluatie van het verwachte resultaat wordt uitgevoerd door middel van een vergelijking met de traditionele, handmatige manieren van code reviews.


% TODO: Trek een duidelijke conclusie, in de vorm van een antwoord op de
% onderzoeksvra(a)g(en). Wat was jouw bijdrage aan het onderzoeksdomein en
% hoe biedt dit meerwaarde aan het vakgebied/doelgroep? 
% Reflecteer kritisch over het resultaat. In Engelse teksten wordt deze sectie
% ``Discussion'' genoemd. Had je deze uitkomst verwacht? Zijn er zaken die nog
% niet duidelijk zijn?
% Heeft het onderzoek geleid tot nieuwe vragen die uitnodigen tot verder 
%onderzoek?

\lipsum[76-80]

