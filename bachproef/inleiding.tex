%%=============================================================================
%% Inleiding
%%=============================================================================

\chapter{\IfLanguageName{dutch}{Inleiding}{Introduction}}%
\label{ch:inleiding}

Vaak wordt een mainframe gezien als verouderde technolgie die enkel gekend is uit de films. Het tegengestelde is waar, mainframes zijn één van de meest geavanceerde servers die telkens onderhouden en gemoderniseerd worden. Mainframes zijn dan ook nog volop in gerbruik. Door de betrouwbaarheid en capaciteit van een mainframe is het vooral geschikt voor grote aantallen transacties/aanvragen van gebruikers. Mainframes worden vaak gebruikt door banken, verzekeraars, vliegmaatschapijen... om hun fundamentele bedrijfsprocessen te ondersteunen. 

Omdat deze systemen zo fundamenteel zijn is het onderhoud ervan van groot belang. Vaak zijn de programma's die draaien op de mainframe ook tientallen jaren oud. Een deel van deze programma's zijn geprogrammeert in High Level Assembler (HLASM). Deze programmeertaal is vaak gekozen omdat het programma snel moet zijn en directe controle nodig heeft tot de hardware. Andere programma's maken vaak gebruik van de HLASM programma's waardoor ze er afhankelijk van worden. 

Daarom is het onderhouden van deze code van groot belang. Daarbij zijn deze programma's vaak miniem gedocumenteerd of bevat het verouderde documentatie. Hierdoor is het moeilijk om als nieuwe mainframe developer de voorgaande code te begrijpen. 

Veel bedrijven hadden het idee dat mainframes onnodig gingen zijn en dat ze konden vertrouwen op ''modernere'' servers. In werkelijkheid groeide hun workload alleen maar en was de workload te groot om op reguliere servers te runnen. Met de vloeiende modernisatie leek de mainframe echter de beste keuze, om verder hun bedrijfsactiviteiten op te vertrouwen. 

Met de mindset dat ze van de mainframe gingen gaan, invisteerde bedrijven niet in nieuw mainframe talent. Hierdoor is de industrie vooral opgemaakt door proffesionals die binnekort volop met pensioen gaan. Nu dat het duidelijk is dat een mainframe beter in staat is om hun huidige workload te beheren, is het tekort aan junior talent een probleem. De kennis van de bedrijfs- programma's en systemen ziten dus in een generatie die de werkracht verlaat. 

Hierdoor is er een nood aan een nieuwe generatie om vooral de onderhoud van de systemen en de programma's te beheren. Door het gebrek aan voldoende en bruikbare documentatie is het vaak moeilijk om als nieuwe mainframe programmeur code te begrijpen. Wat vervolgens het onderhouden ervan moeilijk maakt. Documentatie is dus belangrijk voor de leesbaarheid en onderhoudbaarheid van de code, waardoor minder fouten worden gemaakt.

Een hulpmiddel zou een visualisatie tool zijn die kan dienen als documentatie. Beginners hebben vaak vooral moeite met het begrijpen van de control flow van een programma, daarom is visaulisatie van de control flow vooral handig.

%%-------------------------

%De inleiding moet de lezer net genoeg informatie verschaffen om het onderwerp te begrijpen en in te zien waarom de onderzoeksvraag de moeite waard is om te onderzoeken. In de inleiding ga je literatuurverwijzingen beperken, zodat de tekst vlot leesbaar blijft. Je kan de inleiding verder onderverdelen in secties als dit de tekst verduidelijkt. Zaken die aan bod kunnen komen in de inleiding~\autocite{Pollefliet2011}:

%\begin{itemize}
  %\item context, achtergrond
  %\item afbakenen van het onderwerp
  %\item verantwoording van het onderwerp, methodologie
  %\item probleemstelling
  %\item onderzoeksdoelstelling
  %\item onderzoeksvraag
  %\item \ldots
%\end{itemize}

%\section{\IfLanguageName{dutch}{Probleemstelling}{Problem Statement}}%
%\label{sec:probleemstelling}

%Uit je probleemstelling moet duidelijk zijn dat je onderzoek een meerwaarde heeft voor een concrete doelgroep. De doelgroep moet goed gedefinieerd en afgelijnd zijn. Doelgroepen als ``bedrijven,'' ``KMO's'', systeembeheerders, enz.~zijn nog te vaag. Als je een lijstje kan maken van de personen/organisaties die een meerwaarde zullen vinden in deze bachelorproef (dit is eigenlijk je steekproefkader), dan is dat een indicatie dat de doelgroep goed gedefinieerd is. Dit kan een enkel bedrijf zijn of zelfs één persoon (je co-promotor/opdrachtgever).

%\section{\IfLanguageName{dutch}{Onderzoeksvraag}{Research question}}%
%\label{sec:onderzoeksvraag}

%Wees zo concreet mogelijk bij het formuleren van je onderzoeksvraag. Een onderzoeksvraag is trouwens iets waar nog niemand op dit moment een antwoord heeft (voor zover je kan nagaan). Het opzoeken van bestaande informatie (bv. ``welke tools bestaan er voor deze toepassing?'') is dus geen onderzoeksvraag. Je kan de onderzoeksvraag verder specifiëren in deelvragen. Bv.~als je onderzoek gaat over performantiemetingen, dan 

%\section{\IfLanguageName{dutch}{Onderzoeksdoelstelling}{Research objective}}%
%\label{sec:onderzoeksdoelstelling}

%Wat is het beoogde resultaat van je bachelorproef? Wat zijn de criteria voor succes? Beschrijf die zo concreet mogelijk. Gaat het bv.\ om een proof-of-concept, een prototype, een verslag met aanbevelingen, een vergelijkende studie, enz.

%\section{\IfLanguageName{dutch}{Opzet van deze bachelorproef}{Structure of this bachelor thesis}}%
%\label{sec:opzet-bachelorproef}

%% Het is gebruikelijk aan het einde van de inleiding een overzicht te
%% geven van de opbouw van de rest van de tekst. Deze sectie bevat al een aanzet
%% die je kan aanvullen/aanpassen in functie van je eigen tekst.

%De rest van deze bachelorproef is als volgt opgebouwd:

%In Hoofdstuk~\ref{ch:stand-van-zaken} wordt een overzicht gegeven van de stand van zaken binnen het onderzoeksdomein, op basis van een literatuurstudie.

%In Hoofdstuk~\ref{ch:methodologie} wordt de methodologie toegelicht en worden de gebruikte onderzoekstechnieken besproken om een antwoord te kunnen formuleren op de onderzoeksvragen.

% TODO: Vul hier aan voor je eigen hoofstukken, één of twee zinnen per hoofdstuk

%In Hoofdstuk~\ref{ch:conclusie}, tenslotte, wordt de conclusie gegeven en een antwoord geformuleerd op de onderzoeksvragen. Daarbij wordt ook een aanzet gegeven voor toekomstig onderzoek binnen dit domein.

%%-------------------------