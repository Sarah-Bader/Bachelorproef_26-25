%%=============================================================================
%% Inleiding
%%=============================================================================

\chapter{\IfLanguageName{dutch}{Inleiding}{Introduction}}%
\label{ch:inleiding}

\section{Probleemstelling}
\label{sec:probleemstelling}

Vaak wordt een mainframe gezien als verouderde technologie die enkel gekend is uit films. Het tegengestelde is waar: mainframes zijn één van de meest geavanceerde servers die voortdurend onderhouden en gemoderniseerd worden. Mainframes zijn dan ook nog volop in gebruik. Door de betrouwbaarheid en capaciteit van een mainframe is het vooral geschikt voor grote aantallen transacties/aanvragen van gebruikers. Mainframes worden vaak gebruikt door banken, verzekeraars, luchtvaartmaatschappijen... om hun fundamentele bedrijfsprocessen te ondersteunen.

Omdat deze systemen zo fundamenteel zijn, is het onderhoud ervan van groot belang. Vaak zijn de programma's die draaien op het mainframe ook tientallen jaren oud. Een deel van deze programma's is geprogrammeerd in High Level Assembler (HLASM). Deze programmeertaal wordt vaak gekozen omdat het programma snel moet zijn en directe controle over de hardware nodig heeft. Andere programma's maken vaak gebruik van de HLASM-programma's, waardoor ze ervan afhankelijk worden. 

Daarbij zijn deze programma's vaak minimaal gedocumenteerd of bevatten ze verouderde documentatie. Hierdoor is het moeilijk om als nieuwe mainframe developer de voorgaande code te begrijpen. 

Veel bedrijven hadden het idee dat mainframes onnodig zouden worden en dat ze konden vertrouwen op “modernere” servers. In werkelijkheid groeide hun workload alleen maar en werd die te groot om op reguliere servers te runnen. Met de voortdurende modernisatie bleek het mainframe echter de beste keuze om hun bedrijfsactiviteiten verder op te vertrouwen.  

Met de oude mindset dat ze van het mainframe zouden weggaan, investeerden bedrijven niet in nieuw mainframe talent. Hierdoor bestaat de industrie vooral uit professionals die binnenkort met pensioen gaan. Nu het duidelijk is dat een mainframe beter in staat is om hun huidige workload te beheren, is het tekort aan junior talent een probleem. De kennis van de bedrijfsprogramma's en systemen zit dus in een generatie die de arbeidsmarkt verlaat. 

Hierdoor is er nood aan een nieuwe generatie om vooral het onderhoud van de systemen en programma's te beheren. Door het gebrek aan voldoende en bruikbare documentatie is het vaak moeilijk om als nieuwe mainframe programmeur code te begrijpen, wat het onderhouden ervan belemmert. Documentatie is dus belangrijk voor de leesbaarheid en onderhoudbaarheid van de code, waardoor minder fouten worden gemaakt.

Een hulpmiddel zou een visualisatietool kunnen zijn die kan dienen als documentatie. Beginners hebben vaak moeite met het begrijpen van de control flow van een programma, daarom is visualisatie van de control flow vooral handig.

\section{De onderzoeksvraag}
\label{sec:onderzoeksvraag}

Dit onderzoek is gericht op het beantwoorden van de volgende vraag:

\begin{quote}
    \itshape
    Hoe effectief versnellen statische analyse en Mermaid-diagrammen het leerproces van HLASM bij beginners?
\end{quote}

Binnen dit onderzoek wordt gefocust op twee aspecten: het technische en het menselijke. Er wordt gekeken naar hoe de visualisatie technisch kan geïmplementeerd worden en ook naar de effectiviteit van de visualisatie als documentatie en hulpmiddel bij het begrijpen van de control flow in HLASM.  

\section{Onderzoeksdoelstelling}
\label{sec:Onderzoeksdoelstelling}

Het doel van dit onderzoek is het uitwerken van een Proof-of-Concept en een bijhorende analyse ervan. De PoC moet in staat zijn de belangrijkste onderdelen van HLASM-code om te zetten in een Mermaid.js flowchart. De focus ligt expliciet op de structuur van het programma, dus minder op de details. Het moet bijvoorbeeld weergeven waar het programma begint, waar het zich opsplitst in verschillende takken en waar het eindigt.  

Vooraleer de implementatie van start gaat, wordt onderzoek gedaan naar de bestaande technologieën die van hulp kunnen zijn. Dit onderzoek wordt uitgevoerd in de vorm van een literatuurstudie. Tijdens de literatuurstudie wordt gekeken naar het probleemdomein en het oplossingsdomein. 

Binnen de literatuurstudie worden volgende vragen gesteld om het probleemdomein beter te verstaan:  

\begin{itemize}
    \item Waarom is een mainframe nog steeds van belang? 
    \item Waarom is er nood aan het onderhouden van HLASM-code?
    \item Hoe helpt visualisatie met het begrijpen van codestructuur? 
    \item Wat zijn de moeilijkheden bij het begrijpen van assemblytalen voor beginnende ontwikkelaars?
\end{itemize}

Rond het oplossingsdomein worden volgende vragen bestudeerd:

\begin{itemize}
    \item Hoe helpt visualisatie bij het sneller begrijpen van programma's?
    \item Welke tools bestaan er al om HLASM-code te visualiseren?
    \item Welke HLASM parsers bestaan er?
    \item Wat is de beste manier om de kolom-gebaseerde structuur van HLASM te verwerken met een JavaScript-parser?
    \item Welke JavaScript-libraries zijn toegankelijk en bruikbaar? 
\end{itemize}

\section{Het verwachte resultaat}
\label{sec:resultaat}

Het verwachte resultaat van dit onderzoek is dat de visualisatie helpt bij het beheersen van HLASM-control flow, in tegenstelling tot het gebruik van manuele methodes. Om dit resultaat te meten wordt de PoC geëvalueerd aan de hand van testen die de tijd meten met de PoC en met de manuele manier.

\section{Structuur van het onderzoek}
\label{sec:structuur}

Deze bachelorproef is als volgt opgesteld: 

In Hoofdstuk 2 (Literatuurstudie) worden de deelvragen uit het oplossingsdomein en het probleemdomein onderzocht aan de hand van bestaande onderzoeken en literatuur. Hiermee wordt een beter zicht gecreëerd op het probleem en de oplossing om zo gericht mogelijk te werken. 

In Hoofdstuk 3 (Methodologie) wordt toegelicht welke onderzoekstechnieken worden gebruikt.

In Hoofdstuk 4 (HLASM-analyse) wordt gekeken naar Assembler, hoe de programmeertaal is opgebouwd en welke elementen de structuur van het programma vastleggen. 

In Hoofdstuk 5 (Control-flow visualisatie) wordt bekeken hoe de Assembler-structuur kan worden omgezet naar Mermaid.js-elementen. Er wordt onderzocht hoe dit visueel logisch kan worden gerepresenteerd. 

In Hoofdstuk 6 (Proof-of-Concept) wordt de PoC uitgewerkt en daarna geëvalueerd aan de hand van testen. 

In Hoofdstuk 7 (Conclusie) wordt gekeken naar het resultaat van de evaluatie en naar de literatuurstudie om een antwoord te formuleren op de onderzoeksvragen.  


%%-------------------------

%De inleiding moet de lezer net genoeg informatie verschaffen om het onderwerp te begrijpen en in te zien waarom de onderzoeksvraag de moeite waard is om te onderzoeken. In de inleiding ga je literatuurverwijzingen beperken, zodat de tekst vlot leesbaar blijft. Je kan de inleiding verder onderverdelen in secties als dit de tekst verduidelijkt. Zaken die aan bod kunnen komen in de inleiding~\autocite{Pollefliet2011}:

%\begin{itemize}
  %\item context, achtergrond
  %\item afbakenen van het onderwerp
  %\item verantwoording van het onderwerp, methodologie
  %\item probleemstelling
  %\item onderzoeksdoelstelling
  %\item onderzoeksvraag
  %\item \ldots
%\end{itemize}

%\section{\IfLanguageName{dutch}{Probleemstelling}{Problem Statement}}%
%\label{sec:probleemstelling}

%Uit je probleemstelling moet duidelijk zijn dat je onderzoek een meerwaarde heeft voor een concrete doelgroep. De doelgroep moet goed gedefinieerd en afgelijnd zijn. Doelgroepen als ``bedrijven,'' ``KMO's'', systeembeheerders, enz.~zijn nog te vaag. Als je een lijstje kan maken van de personen/organisaties die een meerwaarde zullen vinden in deze bachelorproef (dit is eigenlijk je steekproefkader), dan is dat een indicatie dat de doelgroep goed gedefinieerd is. Dit kan een enkel bedrijf zijn of zelfs één persoon (je co-promotor/opdrachtgever).

%\section{\IfLanguageName{dutch}{Onderzoeksvraag}{Research question}}%
%\label{sec:onderzoeksvraag}

%Wees zo concreet mogelijk bij het formuleren van je onderzoeksvraag. Een onderzoeksvraag is trouwens iets waar nog niemand op dit moment een antwoord heeft (voor zover je kan nagaan). Het opzoeken van bestaande informatie (bv. ``welke tools bestaan er voor deze toepassing?'') is dus geen onderzoeksvraag. Je kan de onderzoeksvraag verder specifiëren in deelvragen. Bv.~als je onderzoek gaat over performantiemetingen, dan 

%\section{\IfLanguageName{dutch}{Onderzoeksdoelstelling}{Research objective}}%
%\label{sec:onderzoeksdoelstelling}

%Wat is het beoogde resultaat van je bachelorproef? Wat zijn de criteria voor succes? Beschrijf die zo concreet mogelijk. Gaat het bv.\ om een proof-of-concept, een prototype, een verslag met aanbevelingen, een vergelijkende studie, enz.

%\section{\IfLanguageName{dutch}{Opzet van deze bachelorproef}{Structure of this bachelor thesis}}%
%\label{sec:opzet-bachelorproef}

%% Het is gebruikelijk aan het einde van de inleiding een overzicht te
%% geven van de opbouw van de rest van de tekst. Deze sectie bevat al een aanzet
%% die je kan aanvullen/aanpassen in functie van je eigen tekst.

%De rest van deze bachelorproef is als volgt opgebouwd:

%In Hoofdstuk~\ref{ch:stand-van-zaken} wordt een overzicht gegeven van de stand van zaken binnen het onderzoeksdomein, op basis van een literatuurstudie.

%In Hoofdstuk~\ref{ch:methodologie} wordt de methodologie toegelicht en worden de gebruikte onderzoekstechnieken besproken om een antwoord te kunnen formuleren op de onderzoeksvragen.

% TODO: Vul hier aan voor je eigen hoofstukken, één of twee zinnen per hoofdstuk

%In Hoofdstuk~\ref{ch:conclusie}, tenslotte, wordt de conclusie gegeven en een antwoord geformuleerd op de onderzoeksvragen. Daarbij wordt ook een aanzet gegeven voor toekomstig onderzoek binnen dit domein.

%%-------------------------