%%=============================================================================
%% Methodologie
%%=============================================================================

\chapter{\IfLanguageName{dutch}{Methodologie}{Methodology}}%
\label{ch:methodologie}

Binnen dit hoofdstuk wordt toegelicht hoe dit toegepast en technisch onderzoek werd uitgevoerd. Het onderzoeksproces werd uitgevoerd in verschillende fasen. Per fase wordt de doelstelling, de gebruikte onderzoeksmethode en het bijhorende resultaat toegelicht. 


\section{Fase 1: Literatuurstudie}

Binnen de literatuurstudie werd onderzoek gedaan naar het probleemdomein en het oplossingsdomein. De doelstelling was om inzicht te krijgen in het belang van mainframes, de rol van HLASM en de moeilijkheden bij het begrijpen van assemblytalen voor beginnende ontwikkelaars. Daarnaast werd onderzocht welke visualisatietechnieken en bestaande tools gebruikt worden voor het analyseren van control flow. 

De onderzoeksmethode in deze fase bestond uit het analyseren van wetenschappelijke literatuur, technische documentatie en bestaande tools. 

Het resultaat van deze fase is Hoofdstuk \ref{ch:literatuurstudie} (Literatuurstudie), waarin de deelvragen worden beantwoord en de noodzaak van een visualisatietool wordt onderbouwd.

\section{Fase 2: Technische analyse van HLASM}

In deze fase werd de structuur van HLASM-broncode geanalyseerd.  De doelstelling was het identificeren van structurele elementen die de control flow representeren, zoals labels, conditionele vertakkingen en sprongen.

De methode bestond uit een technische analyse van representatieve HLASM-codefragmenten. Hierbij werd specifiek gekeken naar de kolomgebaseerde structuur en naar patronen die gebruikt kunnen worden voor statische analyse.

Het resultaat van deze fase is Hoofdstuk \ref{ch:hlasmanalyse} (HLASM-analyse), waarin wordt vastgelegd welke structurele elementen noodzakelijk zijn voor de visualisatie van de control flow.

\section{Fase 3: Requirementsanalyse}

Op basis van de literatuurstudie en de technische analyse werd een requirementsanalyse opgesteld. De doelstelling was het definiëren van een afgebakende en technisch haalbare scope voor de proof-of-concept.

Volgende aspecten werden bepaald:

\begin{itemize}
    \item Welke subset van HLASM-instructies representeert de control flow? 
    \item Welke Mermaid.js-syntax vertaalt zich naar de juiste visualisatie van de control flow?
    \item Wat zijn de meetbare succescriteria voor de evaluatiefase?
\end{itemize}

Het resultaat van deze fase is Hoofdstuk \ref{ch:cfvisualisatie} (Control-flow visualisatie), waarin een duidelijke afbakening en functionele eisen voor de verdere ontwikkeling en evaluatie staan beschreven. 

\section{Fase 4: Ontwerp en Implementatie}

In deze fase werd de proof-of-concept ontwikkeld. De doelstelling was het automatisch omzetten van een geselecteerde subset HLASM-code naar een control-flowdiagram in Mermaid.js-syntax

De ontwikkeling gebeurde iteratief. Er werd een JavaScript-gebaseerde parser ontworpen die via patroonherkenning instructies en labels detecteert in de HLASM-broncode. Vervolgens werden deze gemapt naar Mermaid.js-elementen.

Het resultaat van deze fase is Hoofdstuk \ref{ch:poc} (Proof-of-Concept), waarin het ontwerp en de implementatie wordt weergegeven. 

\section{Fase 5: Evaluatie}

In de evaluatiefase werd onderzocht of de proof-of-concept effectief bijdraagt aan het sneller en correcter begrijpen van HLASM-control flow. 

Er werd een vergelijkende studie uitgevoerd met 6 tot 10 beginnende mainframeontwikkelaars. De deelnemers analyseerden HLASM-codefragmenten, met of zonder gebruik van de PoC.

Om de correctheid te kunnen meten, werden volgende criteria vastgelegd:

\begin{itemize}
    \item het identificeren van start- en eindpunten,
    \item het aanduiden van alle mogelijke vertakkingen,
    \item het uitleggen van een bepaalde conditionele vertakking.
\end{itemize}

Naast correctheid werden de resultaten ook geëvalueerd op de benodigde tijd om de code te analyseren. 

Het resultaat van deze fase wordt besproken in Hoofdstuk \ref{ch:poc} (Proof-of-Concept) en geïnterpreteerd in Hoofdstuk \ref{ch:conclusie} (Conclusie).

%% TODO: In dit hoofstuk geef je een korte toelichting over hoe je te werk bent
%% gegaan. Verdeel je onderzoek in grote fasen, en licht in elke fase toe wat
%% de doelstelling was, welke deliverables daar uit gekomen zijn, en welke
%% onderzoeksmethoden je daarbij toegepast hebt. Verantwoord waarom je
%% op deze manier te werk gegaan bent.
%% 
%% Voorbeelden van zulke fasen zijn: literatuurstudie, opstellen van een
%% requirements-analyse, opstellen long-list (bij vergelijkende studie),
%% selectie van geschikte tools (bij vergelijkende studie, "short-list"),
%% opzetten testopstelling/PoC, uitvoeren testen en verzamelen
%% van resultaten, analyse van resultaten, ...
%%
%% !!!!! LET OP !!!!!
%%
%% Het is uitdrukkelijk NIET de bedoeling dat je het grootste deel van de corpus
%% van je bachelorproef in dit hoofstuk verwerkt! Dit hoofdstuk is eerder een
%% kort overzicht van je plan van aanpak.
%%
%% Maak voor elke fase (behalve het literatuuronderzoek) een NIEUW HOOFDSTUK aan
%% en geef het een gepaste titel.


