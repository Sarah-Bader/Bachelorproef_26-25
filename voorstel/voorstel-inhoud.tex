%---------- Introduction ---------------------------------------------------------

\section{Inleiding}%
\label{sec:inleiding}
Tegenwoordig spelen mainframecomputers een cruciale rol in de dagelijkse activiteiten van bedrijven in de financiële sector, de gezondheidszorg, verzekeringen en tal van andere publieke en private ondernemingen. Deze bedrijven zijn afhankelijk van mainframecomputers en hebben dan ook de nood om deze zo optimaal mogelijk te onderhouden. De applicaties op de mainframecomputer zijn voornamelijk opgebouwd uit High-Level Assembler (HLASM) en zijn vaak tientallen jaren oud. Aangezien veel ervaren mainframe-specialisten hun pensioen naderen, is er nood aan een nieuwe generatie toegewijde mainframe-ent\-housiastelingen. 

Omdat de broncode vaak minimaal gedocumenteerd is, vormt dit een grote uitdaging voor nieuwe mainframe-ontwikkelaars om deze code te begrijpen en te onderhouden. Door dit alles verhoogt het risico op fouten tijdens onderhoud en wordt de industrie steeds afhankelijker van een beperkt aantal experts, waarvan het aantal is aan het afnemen.
Dit onderzoek wil ondersteuning bieden aan beginnende mainframe-applicatieontwikkelaars binnen organisaties die afhankelijk zijn van optimaal onderhoud van legacy HLASM-broncode. Omdat HLASM-broncode vaak minimaal gedocumenteerd is, moeten ontwikkelaars doorgaans vertrouwen op manuele code reviews om de functionaliteit te achterhalen.

De probleemstelling van dit onderzoek luidt als volgt:
\begin{quote}
\itshape
Op welke manier kan statische analyse van HLASM-broncode worden ingezet om control-flowstructuren te visualiseren in Mermaid.js, zodat beginnende mainframe-ontwikkelaars ondersteund worden in de beheersing van legacycode?
\end{quote}

Het doel van dit onderzoek is het ontwikkelen van een proof-of-concept (PoC) die een representatieve subset van HLASM-opcodes en labels statisch analyseert en deze omzet in een visueel control-flowdiagram in Mermaid.js-syntax. Er wordt vertrokken vanuit een literatuurstudie naar control-flowanalyse, gevolgd door de ontwikkeling en implementatie van de technische oplossing. Ten slotte wordt de oplossing geëvalueerd door middel van een vergelijking met traditionele, manuele code reviews. Het gewenste eindresultaat is een prototype dat kan worden gebruikt als ondersteunende factor bij de beheersing en het onderhoud van de control flow van legacy HLASM-code.


%Waarover zal je bachelorproef gaan? Introduceer het thema en zorg dat volgende zaken zeker duidelijk aanwezig zijn:
%\begin{itemize}
 % \item kaderen thema
 % \item de doelgroep
 % \item de probleemstelling en (centrale) onderzoeksvraag
 % \item de onderzoeksdoelstelling
%\end{itemize}

%Denk er aan: een typische bachelorproef is \textit{toegepast onderzoek}, wat betekent dat je start vanuit een concrete probleemsituatie in bedrijfscontext, een \textbf{casus}. Het is belangrijk om je onderwerp goed af te bakenen: je gaat voor die \textit{ene specifieke probleemsituatie} op zoek naar een goede oplossing, op basis van de huidige kennis in het vakgebied.
%De doelgroep moet ook concreet en duidelijk zijn, dus geen algemene of vaag gedefinieerde groepen zoals \emph{bedrijven}, \emph{developers}, \emph{Vlamingen}, enz. Je richt je in elk geval op it-professionals, een bachelorproef is geen populariserende tekst. Eén specifiek bedrijf (die te maken hebben met een concrete probleemsituatie) is dus beter dan \emph{bedrijven} in het algemeen.
%Formuleer duidelijk de onderzoeksvraag! De begeleiders lezen nog steeds te veel voorstellen waarin we geen onderzoeksvraag terugvinden.
%Schrijf ook iets over de doelstelling. Wat zie je als het concrete eindresultaat van je onderzoek, naast de uitgeschreven scriptie? Is het een proof-of-concept, een rapport met aanbevelingen, \ldots Met welk eindresultaat kan je je bachelorproef als een succes beschouwen?

%---------- Stand van zaken ---------------------------------------------------

\section{Literatuurstudie}%
\label{sec:literatuurstudie}

Hier beschrijf je de \emph{state-of-the-art} rondom je gekozen onderzoeksdomein, d.w.z.\ een inleidende, doorlopende tekst over het onderzoeksdomein van je bachelorproef. Je steunt daarbij heel sterk op de professionele \emph{vakliteratuur}, en niet zozeer op populariserende teksten voor een breed publiek. Wat is de huidige stand van zaken in dit domein, en wat zijn nog eventuele open vragen (die misschien de aanleiding waren tot je onderzoeksvraag!)?

Je mag de titel van deze sectie ook aanpassen (literatuurstudie, stand van zaken, enz.). Zijn er al gelijkaardige onderzoeken gevoerd? Wat concluderen ze? Wat is het verschil met jouw onderzoek?

Verwijs bij elke introductie van een term of bewering over het domein naar de vakliteratuur, bijvoorbeeld~\autocite{Hykes2013}! Denk zeker goed na welke werken je refereert en waarom.

Draag zorg voor correcte literatuurverwijzingen! Een bronvermelding hoort thuis \emph{binnen} de zin waar je je op die bron baseert, dus niet er buiten! Maak meteen een verwijzing als je gebruik maakt van een bron. Doe dit dus \emph{niet} aan het einde van een lange paragraaf. Baseer nooit teveel aansluitende tekst op eenzelfde bron.

Als je informatie over bronnen verzamelt in JabRef, zorg er dan voor dat alle nodige info aanwezig is om de bron terug te vinden (zoals uitvoerig besproken in de lessen Research Methods).

% Voor literatuurverwijzingen zijn er twee belangrijke commando's:
% \autocite{KEY} => (Auteur, jaartal) Gebruik dit als de naam van de auteur
%   geen onderdeel is van de zin.
% \textcite{KEY} => Auteur (jaartal)  Gebruik dit als de auteursnaam wel een
%   functie heeft in de zin (bv. ``Uit onderzoek door Doll & Hill (1954) bleek
%   ...'')

Je mag deze sectie nog verder onderverdelen in subsecties als dit de structuur van de tekst kan verduidelijken.

%---------- Methodologie ------------------------------------------------------
\section{Methodologie}%
\label{sec:methodologie}

Deze bachelorproef is een toegepast en technisch onderzoek waarin een oplossing wordt gezocht voor het onderhoud van legacy HLASM-broncode. Het onderzoek start met een literatuurstudie waarin de control flow van HLASM-broncode wordt bestudeerd en vervolgens wordt een subset van representatieve HLASM-opcodes en labels geselecteerd. Hierna start de ontwikkeling van een technische proof-of-concept (PoC) die de geselecteerde subset visueel omzet in een control-flowchart in Mermaid.js-syntax.

\subsection{Fase 1: Oriëntatie en Literatuurstudie}

Tijdens deze fase wordt gefocust op het aanscherpen van de probleemstelling. Dit gebeurt door middel van een literatuurstudie waarin dieper wordt ingegaan op de leercurve van assemblytalen en waarin voorgaande oplossingen worden onderzocht. Daarnaast wordt gekeken naar HLASM-labels en -opcodes om een representatieve subset te definiëren die later kan worden gebruikt bij het visualiseren van de statische control flow.
Het doel van deze fase is het definiëren van de vereisten voor de technische oplossing. De scope wordt zo nauwkeurig mogelijk afgebakend om een specifieke oplossing te garanderen die voldoet aan de requirements.

Tijdens deze fase worden de volgende aspecten bestudeerd en verzameld:

\begin{itemize}
    \item de uitdagingen bij het beheersen van legacycode voor beginnende ontwikkelaars;
    \item bestaande visualisatietechnieken voor programmacomprehensie;
    \item de meest gebruikte HLASM-opcodes en labels die de control flow beïnvloeden.
\end{itemize}

Het resultaat van deze fase is een theoretisch onderzoek naar de probleemstelling en een afgebakende scope voor de verdere ontwikkeling van de technische oplossing.

\subsection{Fase 2: Implementatie en Evaluatie}

Tijdens de implementatie- en evaluatiefase wordt de proof-of-concept uitgewerkt en iteratief geëvalueerd. De ontwikkeling gebeurt volgens een scrum-werkwijze om de effectiviteit van de PoC te waarborgen. Met behulp van de subset die in de vorige fase werd gedefinieerd, wordt een Python-gebaseerde parser ontwikkeld die HLASM-broncode statisch analyseert. De parser identificeert instructies en labels aan de hand van patroonherkenning in de structuur van HLASM-broncode en zet deze via een mapping om naar Mermaid.js-syntax.

Concreet worden tijdens deze fase de volgende aspecten ontwikkeld:

\begin{itemize}
    \item de logische opbouw van de parser op basis van patroonherkenning in de HLASM\-broncodestructuur;
    \item een mapping van de geselecteerde HLASM-opcodes en labels naar Mermaid.js-elementen.
\end{itemize}

Tijdens de ontwikkeling wordt de PoC iteratief beoordeeld via een vergelijkende studie met traditionele, manuele code reviews. Hierbij worden codefragmenten met beide methoden geanalyseerd door vrijwillige, beginnende mainframe\-applicatieontwikkelaars. Zij worden gevraagd de functionaliteit van HLASM-broncode te bepalen aan de hand van zowel de technische als de traditionele methode. De resultaten worden geanalyseerd op basis van de tijd die nodig was om de functionaliteit van de HLASM-broncode te identificeren.

Deze fase resulteert in een werkende PoC die HLASM-broncode omzet naar Mermaid.js-\-flowcharts. Daarnaast worden evaluatieresultaten verkregen die inzicht geven in de effectiviteit van de technische oplossing.

\subsection{Fase 3: Resultaat en Verdediging}

In de laatste fase worden de onderzoeksresultaten verwerkt tot een conclusie. Specifiek worden de volgende aspecten besproken:

\begin{itemize}
    \item de sterktes en beperkingen van de PoC;
    \item mogelijke uitbreidingen voor toekomstig onderzoek.
\end{itemize}

Deze fase resulteert in de finale scriptie en een gedocumenteerde proof-of-concept.

\subsection{Tijdsplanning}

Elke fase heeft een afgebakende tijdsperiode waarin de specifieke resultaten moeten worden opgeleverd voor de volgende fase. Zoals vermeld in figuur \ref{fig:gantt-diagram}, zijn ook de administratieve deliverables opgenomen om samenhang te tonen met de feedback van de promotor.

\begin{figure}[htbp]
    \centering
    \includegraphics[width=\columnwidth]{gantt-diagram.png}
    \caption{Gantt-diagram met een tijdsindicatie per onderzoeksfase.}
    \label{fig:gantt-diagram}
\end{figure}


%---------- Verwachte resultaten ----------------------------------------------
\section{Verwacht resultaat, conclusie}%
\label{sec:verwachte_resultaten}

Het verwachte resultaat van deze studie is een technische oplossing, proof of concept (PoC), die in staat is een subset van gedefinieerde en representatieve HLASM-opcodes en labels om te zetten in visuele stappen. De PoC brengt door middel van statische HLASM-broncodeanalyse de control flow in kaart. Het eindresultaat wordt vertaald naar Mermaid.js-syntax, waarmee een visuele flowchart kan worden gegenereerd.
Met deze PoC wordt verwacht beginnende mainframe\-applicatieontwikkelaars te ondersteunen met het beheersen van een onbekend stuk HLASM-broncode. De evaluatie van het verwachte resultaat wordt uitgevoerd door middel van een vergelijking met de traditionele, manuele manieren van code reviews.




