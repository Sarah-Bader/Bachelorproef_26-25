%==============================================================================
% Sjabloon onderzoeksvoorstel bachproef
%==============================================================================
% Gebaseerd op document class `hogent-article'
% zie <https://github.com/HoGentTIN/latex-hogent-article>

% Voor een voorstel in het Engels: voeg de documentclass-optie [english] toe.
% Let op: kan enkel na toestemming van de bachelorproefcoördinator!
\documentclass{hogent-article}

% Invoegen bibliografiebestand
\addbibresource{voorstel.bib}
 
% Informatie over de opleiding, het vak en soort opdracht
\studyprogramme{Professionele bachelor toegepaste informatica}
\course{Bachelorproef}
\assignmenttype{Onderzoeksvoorstel}

\academicyear{2025-2026} 

% TODO: Werktitel
\title{Statische analyse en visualisatie van control flow in HLASM-legacycode met Mermaid.js, ter ondersteuning van de beheersbaarheid voor beginnende mainframeontwikkelaars.}


\author{Sarah Bader}
\email{sarah.bader@student.hogent.be}


% TODO: Geef de co-promotor op
\supervisor[Co-promotor]{J. Doe (Acme, \href{mailto:John.Doe@mail.com}{John.Doe@mail.com})}

% Binnen welke specialisatierichting uit 3TI situeert dit onderzoek zich?
% Kies uit deze lijst:
%
% - Mobile \& Enterprise development
% - AI \& Data Engineering
% - Functional \& Business Analysis
% - System \& Network Administrator
% - Mainframe Expert
% - Als het onderzoek niet past binnen een van deze domeinen specifieer je deze
%   zelf
%
\specialisation{Mainframe Expert}
\keywords{HLASM, Control Flow Graphs, Static Analysis, Program Visualization, Legacy Systems, z/Architecture, z/OS, Mermaid.js}


\begin{document}

% TODO: Review + Rewrite
\begin{abstract}
    
Mainframeapplicaties zijn voornamelijk opgebouwd uit High-Level Assembler (HLASM), waardoor het onderhoud van deze broncode van groot belang is. Voor beginnende mainframe-ontwikkelaars is deze legacycode echter vaak moeilijk beheersbaar. Daarnaast naderen veel mainframe-specialisten hun pensioen en is deze legacycode vaak miniem gedocumenteerd. Hierdoor wordt het zeer moeilijk en tijdrovend om voortdurend onderhoud aan deze legacycode te waarborgen.
Dit onderzoek richt zich dan ook op het ontwikkelen en ondersteunen van HLASM-legacycode-documentatie door middel van een technische oplossing die zich focust op statische analyse. De proof of concept (PoC) richt zich expliciet op een subset van de belangrijkste HLASM-opcodes die de control flow bepalen. Concreet wordt een op Python gebaseerde parser ontwikkeld die gedefineerde HLASM-opcodes en labels detecteert en deze omzet naar een control-flowdiagram in Mermaid.js-syntax.
De methodologie bestaat uit een literatuurstudie naar statische control-flowanalyse, gevolgd door de evaluatie van de technische oplossing. De PoC wordt vergeleken met traditionele, manuele code reviews manieren om daaruit de nodige conclusies te trekken. Het verwachte resultaat is dat de PoC sneller inzicht biedt in de structuur van HLASM-programma’s en hiermee beginnende mainframe-applicatieontwikkelaars ondersteunt bij het beheersen van bestaande HLASM-broncode.

  
\end{abstract}

\tableofcontents

% De hoofdtekst van het voorstel zit in een apart bestand, zodat het makkelijk
% kan opgenomen worden in de bijlagen van de bachelorproef zelf.
%---------- Introduction ---------------------------------------------------------

\section{Inleiding}%
\label{sec:inleiding}
Tegenwoordig spelen mainframecomputers een cruciale rol in de dagelijkse activiteiten van bedrijven in de financiële sector, de gezondheidszorg, verzekeringen en tal van andere publieke en private ondernemingen. Deze bedrijven zijn afhankelijk van mainframecomputers en hebben dan ook de nood om deze zo optimaal mogelijk te onderhouden. Vermits veel ervaren mainframe-specialisten hun pensioen naderen, is er nood aan een nieuwe generatie mainframe-specialisten die het onderhoud van hun systemen kan waarborgen. Onder dit onderhoud vallen de bedrijfskritische functionaliteiten van mainframeapplicaties, die voornamelijk opgebouwd zijn uit High-Level Assembler (HLASM). Deze legacycode is vaak tientallen jaren oud en minimaal gedocumenteerd. Voor de nieuwe generatie mainframeontwikkelaars vormt dit een uitdaging om deze legacycode te begrijpen en te onderhouden. Zij vertrouwen vaak enkel op manuele code reviews.

Door middel van een geautomatiseerde tool die statische analyse van HLASM-code visualiseert in een control-flowchart, wil dit onderzoek ondersteuning bieden aan beginnende mainframeontwikkelaars, specifiek binnen organisaties die afhankelijk zijn van legacy HLASM-code.

De probleemstelling van dit onderzoek luidt als volgt:
\begin{quote}
    \itshape
    Op welke manier kan statische analyse van HLASM-broncode worden ingezet om control-flowstructuren te visualiseren in Mermaid.js, zodat beginnende mainframeontwikkelaars ondersteund worden in het begrijpen van legacycode?
\end{quote}

Het doel van dit onderzoek is het ontwikkelen van een proof-of-concept (PoC) die een representatieve subset van HLASM-opcodes en labels statisch analyseert en deze omzet in een visueel control-flowdiagram in Mermaid.js-syntax.

Vooraleer de ontwikkeling van de PoC van start gaat, wordt een literatuurstudie uitgevoerd naar het probleemdomein en het oplossingsdomein.

Rond het probleemdomein worden volgende vragen bestudeerd:
\begin{itemize}
    \item Wat is het belang van de mainframecomputer?
    \item Waarom is er nood aan het onderhouden van HLASM-code?
    \item Wat zijn de moeilijkheden bij het begrijpen van assemblytalen voor beginnende ontwikkelaars?
\end{itemize}

Rond het oplossingsdomein worden volgende vragen bestudeerd:
\begin{itemize}
    \item Wat is de effectiviteit van visualisatietechnieken voor programmacomprehensie?
    \item Wat zijn de bestaande visualisatietools binnen High Level Assembler?
    \item Welke HLASM-opcodes en labels bepalen de control flow in een programma?
\end{itemize}


%Waarover zal je bachelorproef gaan? Introduceer het thema en zorg dat volgende zaken zeker duidelijk aanwezig zijn:
%\begin{itemize}
 % \item kaderen thema
 % \item de doelgroep
 % \item de probleemstelling en (centrale) onderzoeksvraag
 % \item de onderzoeksdoelstelling
%\end{itemize}

%Denk er aan: een typische bachelorproef is \textit{toegepast onderzoek}, wat betekent dat je start vanuit een concrete probleemsituatie in bedrijfscontext, een \textbf{casus}. Het is belangrijk om je onderwerp goed af te bakenen: je gaat voor die \textit{ene specifieke probleemsituatie} op zoek naar een goede oplossing, op basis van de huidige kennis in het vakgebied.
%De doelgroep moet ook concreet en duidelijk zijn, dus geen algemene of vaag gedefinieerde groepen zoals \emph{bedrijven}, \emph{developers}, \emph{Vlamingen}, enz. Je richt je in elk geval op it-professionals, een bachelorproef is geen populariserende tekst. Eén specifiek bedrijf (die te maken hebben met een concrete probleemsituatie) is dus beter dan \emph{bedrijven} in het algemeen.
%Formuleer duidelijk de onderzoeksvraag! De begeleiders lezen nog steeds te veel voorstellen waarin we geen onderzoeksvraag terugvinden.
%Schrijf ook iets over de doelstelling. Wat zie je als het concrete eindresultaat van je onderzoek, naast de uitgeschreven scriptie? Is het een proof-of-concept, een rapport met aanbevelingen, \ldots Met welk eindresultaat kan je je bachelorproef als een succes beschouwen?

%---------- Stand van zaken ---------------------------------------------------

\section{Literatuurstudie}%
\label{sec:literatuurstudie}

%  TODO : herschrijven... 

\subsection{Belang van de mainframecomputer}

Mainframes hebben de reputatie verouderde technologie te zijn en voor een lange tijd werd gepleit dat de mainframecomputer zou worden vervangen. Momenteel blijven mainframes bedrijfskritische applicaties draaien en worden ze geïntegreerd met moderne technologieën \autocite{Sagers_Ball_Hosack_Twitchell_Wallace_2018}. Het onderzoek van \textcite{Sagers_Ball_Hosack_Twitchell_Wallace_2018} wordt bevestigd door recent onderzoek waarin wordt aangetoond dat organisaties niet enkel gebruikmaken van legacyapplicaties, maar ook bezig zijn met het actief ontwikkelen van nieuwe applicaties \autocite{shaik_2024}.

Ondanks het feit dat mainframes zeer relevant en modern zijn, is er een dalend aantal ontwikkelaars met mainframe-kennis. De ervaren professionals naderen hun pensioen en er is een beperkt aanbod aan mainframegerichte opleidingen. Dit veroorzaakt een probleem voor het toekomstige onderhoud van mainframes \autocite{waites_ketterer_2013}. Er is dus een nood aan een nieuwe generatie mainframespecialisten die dit onderhoud kan waarborgen.

\subsection{Nood aan onderhoud van HLASM-code}

Bedrijfskritische functionaliteiten van mainframeapplicaties zijn vaak geïmplementeerd in High Level Assembler (HLASM), waardoor onderhoud van deze legacycode van groot belang is. Uit een onderzoek naar softwareonderhoud blijkt dat het begrijpen van bestaande code een groot deel van de moeite neemt tijdens onderhoud. Vooral bij beginnende ontwikkelaars gaat het merendeels van hun tijd aan het begrijpen van de functionaliteit, dit is omdat zij minder ervaring hebben \autocite{xia_bao_lo_xing_hassan_li_2018}. Dit is vooral het geval bij legacycode zoals HLASM, omdat daar de documentatie vaak beperkt is. Voor dit probleem biedt een code control-flow visualisatie tool meer en makkelijker inzicht in de fuctionaliteiten van broncode. 

\subsection{Moeilijkheden voor beginnende ontwikkelaars}

Assemblytalen zijn zeer gedetailleerd. Als ontwikkelaar definieer je elke kleine stap van het programma, wat ingewikkeld kan zijn voor beginners. Uit onderzoek blijkt dat naarmate de programmeerervaring toeneemt, het gemakkelijker wordt om de functionaliteit van code te begrijpen \autocite{feigenspan_kästner_liebig_apel_hanenberg_2012}.

Binnen het onderwijs hebben studenten vooral moeite met de componenten die de logische volgorde van het programma bepalen. Dit zijn de componenten die de control flow van het programma definiëren. Bij het doorgronden hoe de control flow loopt, maken beginners regelmatig fouten en ervaren ze vaak verwarring \autocite{kawash_2024}. Hiermee kan worden geconcludeerd dat een visuele representatie kan bijdragen aan het verkrijgen van inzicht in de control flow.

\subsection{Effectiviteit van visualisatietechnieken}

Visualisatietechnieken worden vaak gebruikt om code beter te begrijpen en worden ook erkend als technieken die sneller en meer inzicht geven in codefunctionaliteiten. Visualisaties worden vaak gebruikt om de structuur van code te representeren \autocite{storey_2005}. Bij assemblytalen kan visualisatie vooral helpen, aangezien de code vaak is opgebouwd uit zeer gedetailleerde instructies en kan worden gegroepeerd in de belangrijkste componenten \autocite{baldwin_myers_storey_coady_2009}.

\subsection{Bestaande visualisatietools}

Een bestaande tool die kan worden ingezet voor de visualisatie van HLASM-code is FermaT. Deze tool is voornamelijk gericht op het begrijpen en migreren van legacycode. Het is een zeer krachtige, maar ook zware tool, die vooral focust op codeconversies en diepgaande analyses \autocite{ward_2001}.

Het doel van dit onderzoek is het ontwikkelen van een lichte tool die HLASM-code omzet naar high-level control-flowdiagrammen in Mermaid.js-syntax. Deze tool is specifiek bedoeld voor educatieve doeleinden, met als doel beginnende ontwikkelaars te ondersteunen. In tegenstelling tot FermaT ligt de focus uitsluitend op een heldere en toegankelijke visualisatie van de control flow.



\subsection{High Level Assembler}
High Level Assembler (HLASM) wordt beschouwd als een Second Generation Language (2GL), de instructies in HLASM komen rechtstreeks overeen met processorcommando's. Er zijn dus geen abstracties zoals functies of objecten. Deze abstracties zijn wel terug te vinden in talen met een hoger generatieniveau zoals COBOL. HLASM biedt daardoor veel controle, maar dat maakt de code vaak moeilijker te lezen en te onderhouden \autocite{kornelis_2003}.

Een HLASM-programma bestaat uit één of meerdere modules. Elke module wordt samengesteld uit één bronbestand, de bijbehorende copybooks en macro's. De structuur van HLASM-code wordt bepaald door opcodes en labels. Om een subroutine aan te roepen, kun je gebruikmaken van opcodes zoals BAL of BAS. Deze opcodes slaan het retouradres (het adres van de instructie na de call) op in een register. Om terug te keren vanuit een subroutine kun je gebruikmaken van de opcode BR (Branch to Register), met deze opcode spring je naar het adres in de meegegeven operand (het register) \autocite{ward_2013}.

Op de IBM 370- en z/OS-mainframes is er geen hardware-stack. In plaats daarvan wordt er gebruikgemaakt van een gelinkte lijst van save areas. Een save area is opslagruimte waarin kopieën van alle registerwaarden en de verwijzingen naar de volgende en vorige save area zijn opgeslagen. De control flow is dus gedefinieerd in de registers en niet in een stack. Hierdoor is het de taak van de ontwikkelaar om de calls en de returns van en naar subroutines zelf goed af te handelen door middel van de registerwaarden \autocite{ward_2013}.

HLASM-code heeft vaste labels en opcodes om de control flow van het programma te bepalen. Voor het aanroepen heb je opcodes zoals BAL of BAS en voor het terugkeren heb je bijvoorbeeld BR, hierdoor kan de structuur van het programma vastgesteld worden. Dit biedt de mogelijkheid om op een geautomatiseerde manier HLASM-code statisch te analyseren en te visualiseren naar een control-flowchart in Mermaid.js-syntax.


%Hier beschrijf je de \emph{state-of-the-art} rondom je gekozen onderzoeksdomein, d.w.z.\ een inleidende, doorlopende tekst over het onderzoeksdomein van je bachelorproef. Je steunt daarbij heel sterk op de professionele \emph{vakliteratuur}, en niet zozeer op populariserende teksten voor een breed publiek. Wat is de huidige stand van zaken in dit domein, en wat zijn nog eventuele open vragen (die misschien de aanleiding waren tot je onderzoeksvraag!)? Je mag de titel van deze sectie ook aanpassen (literatuurstudie, stand van zaken, enz.). Zijn er al gelijkaardige onderzoeken gevoerd? Wat concluderen ze? Wat is het verschil met jouw onderzoek? Verwijs bij elke introductie van een term of bewering over het domein naar de vakliteratuur, bijvoorbeeld~\autocite{Hykes2013}! Denk zeker goed na welke werken je refereert en waarom. Draag zorg voor correcte literatuurverwijzingen! Een bronvermelding hoort thuis \emph{binnen} de zin waar je je op die bron baseert, dus niet er buiten! Maak meteen een verwijzing als je gebruik maakt van een bron. Doe dit dus \emph{niet} aan het einde van een lange paragraaf. Baseer nooit teveel aansluitende tekst op eenzelfde bron. Als je informatie over bronnen verzamelt in JabRef, zorg er dan voor dat alle nodige info aanwezig is om de bron terug te vinden (zoals uitvoerig besproken in de lessen Research Methods).

% Voor literatuurverwijzingen zijn er twee belangrijke commando's:
% \autocite{KEY} => (Auteur, jaartal) Gebruik dit als de naam van de auteur
%   geen onderdeel is van de zin.
% \textcite{KEY} => Auteur (jaartal)  Gebruik dit als de auteursnaam wel een
%   functie heeft in de zin (bv. ``Uit onderzoek door Doll & Hill (1954) bleek
%   ...'')

%Je mag deze sectie nog verder onderverdelen in subsecties als dit de structuur van de tekst kan verduidelijken.

%---------- Methodologie ------------------------------------------------------
\section{Methodologie}%
\label{sec:methodologie}

Deze bachelorproef is een toegepast en technisch onderzoek waarin een oplossing wordt gezocht voor het onderhoud van legacy HLASM-broncode. Het onderzoek start met een literatuurstudie waarin de control flow van HLASM-broncode wordt bestudeerd en vervolgens wordt een subset van representatieve HLASM-opcodes en labels geselecteerd. Hierna start de ontwikkeling van een technische proof-of-concept (PoC) die de geselecteerde subset visueel omzet in een control-flowchart in Mermaid.js-syntax.

\subsection{Fase 1: Oriëntatie en Literatuurstudie}

Tijdens deze fase wordt gefocust op het aanscherpen van de probleemstelling. Dit gebeurt door middel van een literatuurstudie waarin dieper wordt ingegaan op de leercurve van assemblytalen en waarin voorgaande oplossingen worden onderzocht. Daarnaast wordt gekeken naar HLASM-labels en -opcodes om een representatieve subset te definiëren die later kan worden gebruikt bij het visualiseren van de statische control flow.
Het doel van deze fase is het definiëren van de vereisten voor de technische oplossing. De scope wordt zo nauwkeurig mogelijk afgebakend om een specifieke oplossing te garanderen die voldoet aan de requirements.

Tijdens deze fase worden de volgende aspecten bestudeerd en verzameld:

\begin{itemize}
    \item de uitdagingen bij het beheersen van legacycode voor beginnende ontwikkelaars;
    \item bestaande visualisatietechnieken voor programmacomprehensie;
    \item de meest gebruikte HLASM-opcodes en labels die de control flow beïnvloeden.
\end{itemize}

Het resultaat van deze fase is een theoretisch onderzoek naar de probleemstelling en een afgebakende scope voor de verdere ontwikkeling van de technische oplossing.

\subsection{Fase 2: Implementatie en Evaluatie}

Tijdens de implementatie- en evaluatiefase wordt de proof-of-concept uitgewerkt en iteratief geëvalueerd. De ontwikkeling gebeurt volgens een scrum-werkwijze om de effectiviteit van de PoC te waarborgen. Met behulp van de subset die in de vorige fase werd gedefinieerd, wordt een Python-gebaseerde parser ontwikkeld die HLASM-broncode statisch analyseert. De parser identificeert instructies en labels aan de hand van patroonherkenning in de structuur van HLASM-broncode en zet deze via een mapping om naar Mermaid.js-syntax.

Concreet worden tijdens deze fase de volgende aspecten ontwikkeld:

\begin{itemize}
    \item de logische opbouw van de parser op basis van patroonherkenning in de HLASM\-broncodestructuur;
    \item een mapping van de geselecteerde HLASM-opcodes en labels naar Mermaid.js-elementen.
\end{itemize}

Tijdens de ontwikkeling wordt de PoC iteratief beoordeeld via een vergelijkende studie met traditionele, manuele code reviews. Hierbij worden codefragmenten met beide methoden geanalyseerd door vrijwillige, beginnende mainframe\-applicatieontwikkelaars. Zij worden gevraagd de functionaliteit van HLASM-broncode te bepalen aan de hand van zowel de technische als de traditionele methode. De resultaten worden geanalyseerd op basis van de tijd die nodig was om de functionaliteit van de HLASM-broncode te identificeren.

Deze fase resulteert in een werkende PoC die HLASM-broncode omzet naar Mermaid.js-\-flowcharts. Daarnaast worden evaluatieresultaten verkregen die inzicht geven in de effectiviteit van de technische oplossing.

\subsection{Fase 3: Resultaat en Verdediging}

In de laatste fase worden de onderzoeksresultaten verwerkt tot een conclusie. Specifiek worden de volgende aspecten besproken:

\begin{itemize}
    \item de sterktes en beperkingen van de PoC;
    \item mogelijke uitbreidingen voor toekomstig onderzoek.
\end{itemize}

Deze fase resulteert in de finale scriptie en een gedocumenteerde proof-of-concept.

\subsection{Tijdsplanning}

Elke fase heeft een afgebakende tijdsperiode waarin de specifieke resultaten moeten worden opgeleverd voor de volgende fase. Zoals vermeld in figuur \ref{fig:gantt-diagram}, zijn ook de administratieve deliverables opgenomen om samenhang te tonen met de feedback van de promotor.

\begin{figure}[htbp]
    \centering
    \includegraphics[width=\columnwidth]{gantt-diagram.png}
    \caption{Gantt-diagram met een tijdsindicatie per onderzoeksfase.}
    \label{fig:gantt-diagram}
\end{figure}


%---------- Verwachte resultaten ----------------------------------------------
\section{Verwacht resultaat, conclusie}%
\label{sec:verwachte_resultaten}

Het verwachte resultaat van deze studie is een proof of concept (PoC), die in staat is een subset van gedefinieerde en representatieve HLASM-opcodes en labels om te zetten in visuele stappen. De PoC brengt door middel van statische HLASM-broncodeanalyse de control flow in kaart. Het eindresultaat wordt vertaald naar Mermaid.js-syntax, waarmee een visuele flowchart kan worden gegenereerd.
Met deze PoC wordt verwacht beginnende mainframeontwikkelaars te ondersteunen met het begrijpen van een onbekend stuk HLASM-broncode. De evaluatie van het verwachte resultaat wordt uitgevoerd door middel van een vergelijking met de traditionele, manuele manieren van code reviews.






\printbibliography[heading=bibintoc]

\end{document}